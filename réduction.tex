\documentclass[12pt,a4paper,titlepage]{article}

\usepackage[utf8]{inputenc}
\usepackage[T1]{fontenc}
\usepackage[french]{babel}
\usepackage{amsmath, amssymb}
\usepackage{microtype}
\usepackage{hyperref}
\usepackage{geometry}
\geometry{margin=2.5cm}

\title{Informatique Fondamentale \\ Réduction vers SAT du problème des traversées de poules}
\author{CORMONTAGNE Romain\and GODRIE Clément\and KHARRAZ HAMOUCH Chakir}
\date{21 Décembre 2025}

\begin{document}
\maketitle

\section{Rappel du problème}

On considère un ensemble de $N$ poules initialement situées sur une berge $A$.
Chaque poule $p$ possède une durée de traversée propre $T_p$.
Une barque de capacité maximale $C$ permet de transporter des poules entre les deux berges $A$ et $B$.
À chaque traversée, la durée du voyage est égale au maximum des durées des poules embarquées.

L’objectif est de déterminer s’il existe un plan de traversées permettant de faire passer toutes les poules sur la berge $B$ en un temps total inférieur ou égal à une borne $T$.

Une réduction de ce problème vers SAT, implémentée à l’aide de la librairie \texttt{python-sat}, est présentée dans ce rapport.

\section{Variables propositionnelles}
On note $T_N = \max_{p \in \{1,\dots,N\}} T_p$ la durée maximale de traversée parmi toutes les poules.
Pour $p \in \{1,\dots,N\}$, $t \in \{0,\dots,T\}$ et $d \in \{0,\dots,T_N\}$, les variables suivantes ont été~introduites :

\begin{itemize}
  \item $A_{p,t}$ : vraie ssi la poule $p$ est sur la berge $A$ à l’instant $t$.
  \item $side_t$ : vraie ssi la barque est sur la berge $A$ à l’instant $t$.
  \item $DEP_t$ : vraie ssi un départ a lieu à l’instant $t$.
  \item $dep_{t,p}$ : vraie ssi la poule $p$ embarque au départ de l’instant $t$.
  \item $dur_{t,d}$ : vraie ssi un voyage de durée $d$ commence à l’instant $t$.
  \item $ARR_t$ : vraie ssi une arrivée a lieu à l’instant $t$.
  \item $move_{t,p}$ : vraie ssi la poule $p$ débarque à l’instant $t$.
  \item $ALL_t$ : vraie ssi toutes les poules sont sur la berge $B$ à l’instant $t$.

\end{itemize}

Nous utilisons également des variables auxiliaires :
\[
link_{t,d,p} \;\leftrightarrow\; dep_{t,p} \wedge dur_{t,d}
\]
qui permettent de relier un embarquement à son arrivée correspondante.

\section{Définition des contraintes}
\subsection*{Initialisation}

Au temps $t=0$, toutes les poules sont sur la berge $A$ et la barque est également sur $A$ :
\[
A_{p,0} \quad\text{et}\quad side_0
\]

\subsection*{Condition d’objectif}

L’objectif est d’atteindre un état où toutes les poules sont sur la berge $B$ à l’instant $T$ :
\[
ALL_T
\]

La variable $ALL_t$ est définie par l’équivalence :
\[
ALL_t \;\leftrightarrow\; \bigwedge_{p=1}^N \neg A_{p,t}
\]

Qui se traduit en CNF par deux séries de clauses :
\[
\bigwedge_{t \in \{0,\dots,T\}}\bigwedge_{p \in \{1,\dots,N\}} \neg ALL_t\; \vee\; \neg A_{p,t} 
\]

et

\[
\bigwedge_{t \in \{0,\dots,T\}}(\bigvee_{p\in\{1;\dots,N\}}A_{p,t}\; \vee\; ALL_t)
\]

\subsection*{Départs et capacité}

Un départ à l’instant $t$ a lieu si et seulement si au moins une poule embarque :
\[
DEP(t) \;\leftrightarrow\; \bigvee_{p=1}^N dep(t,p)
\]

La capacité de la barque est bornée par $c$ à l’aide d’une contrainte de cardinalité :
\[
\sum_{p=1}^N dep(t,p) \le c
\]

Aucun départ n’est autorisé à l’instant final $T$.

\section{Durées des voyages}

Un point central du modèle est que la durée est une \emph{fonction totale du temps}.
À chaque instant $t$, exactement une valeur $dur(t,d)$ est vraie pour un $d \in \{0,\dots,\max T_p\}$.

La valeur $d=0$ joue le rôle de sentinelle et représente l’absence de voyage :
\[
dur(t,0) \;\leftrightarrow\; \neg DEP(t)
\]

Cette modélisation garantit que les instants sans départ sont explicitement représentés, ce qui est essentiel pour la cohérence temporelle du modèle.

Les durées qui dépasseraient l’horizon temporel sont interdites :
\[
t + d > T \;\Rightarrow\; \neg dur(t,d)
\]

\section{Durée d’un voyage}

Si un voyage commence à l’instant $t$ avec une durée $d>0$, alors :
\begin{itemize}
  \item aucune poule embarquée ne peut avoir une durée strictement supérieure à $d$ ;
  \item au moins une poule embarquée doit avoir une durée exactement égale à $d$.
\end{itemize}

Ces contraintes garantissent que la durée du voyage est bien le maximum des durées des poules embarquées.

\section{Cohérence des berges}

Une poule ne peut embarquer que depuis la berge où se trouve la barque :
\[
\begin{cases}
side(t) \land dep(t,p) \Rightarrow A(p,t) \\
\neg side(t) \land dep(t,p) \Rightarrow \neg A(p,t)
\end{cases}
\]

La position de la barque évolue uniquement lors des arrivées :
\begin{itemize}
  \item si $ARR(t)$, alors la barque change de berge ;
  \item sinon, elle reste sur la même berge.
\end{itemize}
Le type du mouvement (aller ou retour) est implicitement déterminé par la position de la barque au moment du départ.


\section{Arrivées et chronologie}

Une arrivée à l’instant $t$ a lieu si et seulement s’il existe un instant $t_0 < t$ et une durée $d$ tels que :
\[
dur(t_0,d) \land t_0 + d = t
\]

Afin de correspondre aux attentes des tests, une \emph{timeline compacte} est imposée :
\begin{itemize}
  \item tout départ $t>0$ doit correspondre à une arrivée ;
  \item après une arrivée, si toutes les poules ne sont pas encore sur $B$, un nouveau départ a lieu immédiatement.
\end{itemize}

\section{Mouvements des poules}

La variable $move(t,p)$ est vraie si et seulement si la poule $p$ était embarquée lors du départ dont l’arrivée a lieu à l’instant $t$.

L’évolution de la position des poules est alors définie par :
\begin{itemize}
  \item sans arrivée, les poules restent sur la même berge ;
  \item avec arrivée, une poule change de berge si et seulement si $move(t,p)$ est vraie.
\end{itemize}

\section{Résolution et reconstruction}

La formule CNF est résolue à l’aide du solveur \texttt{Minisat22}.
À partir du modèle satisfaisant, la solution est reconstruite en listant, pour chaque instant $t$, les poules embarquées lorsque $DEP(t)$ est vraie.

\section{Complexité}

Le nombre de variables et de clauses de la formule SAT produite est polynomial en fonction de $N$, $T$ et $T_N$.
Les indices temporels sont bornés par $T$, supposé donné en base unaire, ce qui garantit que la taille de la réduction est polynomiale.

La réduction du problème des poules contraint vers SAT est donc calculable en temps polynomial.

\section{Conclusion}

Cette réduction vers SAT est fidèle au problème initial et permet de déterminer l’existence d’un plan de traversées dans un temps borné.
Un point clé du modèle est le traitement de la durée comme une fonction totale du temps, ce qui justifie l’introduction de la valeur sentinelle $d=0$.
Les tests fournis confirment la correction et la complétude de l’encodage.

\end{document}
